% Options for packages loaded elsewhere
\PassOptionsToPackage{unicode}{hyperref}
\PassOptionsToPackage{hyphens}{url}
%
\documentclass[
]{article}
\usepackage{amsmath,amssymb}
\usepackage{iftex}
\ifPDFTeX
  \usepackage[T1]{fontenc}
  \usepackage[utf8]{inputenc}
  \usepackage{textcomp} % provide euro and other symbols
\else % if luatex or xetex
  \usepackage{unicode-math} % this also loads fontspec
  \defaultfontfeatures{Scale=MatchLowercase}
  \defaultfontfeatures[\rmfamily]{Ligatures=TeX,Scale=1}
\fi
\usepackage{lmodern}
\ifPDFTeX\else
  % xetex/luatex font selection
\fi
% Use upquote if available, for straight quotes in verbatim environments
\IfFileExists{upquote.sty}{\usepackage{upquote}}{}
\IfFileExists{microtype.sty}{% use microtype if available
  \usepackage[]{microtype}
  \UseMicrotypeSet[protrusion]{basicmath} % disable protrusion for tt fonts
}{}
\makeatletter
\@ifundefined{KOMAClassName}{% if non-KOMA class
  \IfFileExists{parskip.sty}{%
    \usepackage{parskip}
  }{% else
    \setlength{\parindent}{0pt}
    \setlength{\parskip}{6pt plus 2pt minus 1pt}}
}{% if KOMA class
  \KOMAoptions{parskip=half}}
\makeatother
\usepackage{xcolor}
\usepackage[margin=1in]{geometry}
\usepackage{color}
\usepackage{fancyvrb}
\newcommand{\VerbBar}{|}
\newcommand{\VERB}{\Verb[commandchars=\\\{\}]}
\DefineVerbatimEnvironment{Highlighting}{Verbatim}{commandchars=\\\{\}}
% Add ',fontsize=\small' for more characters per line
\usepackage{framed}
\definecolor{shadecolor}{RGB}{248,248,248}
\newenvironment{Shaded}{\begin{snugshade}}{\end{snugshade}}
\newcommand{\AlertTok}[1]{\textcolor[rgb]{0.94,0.16,0.16}{#1}}
\newcommand{\AnnotationTok}[1]{\textcolor[rgb]{0.56,0.35,0.01}{\textbf{\textit{#1}}}}
\newcommand{\AttributeTok}[1]{\textcolor[rgb]{0.13,0.29,0.53}{#1}}
\newcommand{\BaseNTok}[1]{\textcolor[rgb]{0.00,0.00,0.81}{#1}}
\newcommand{\BuiltInTok}[1]{#1}
\newcommand{\CharTok}[1]{\textcolor[rgb]{0.31,0.60,0.02}{#1}}
\newcommand{\CommentTok}[1]{\textcolor[rgb]{0.56,0.35,0.01}{\textit{#1}}}
\newcommand{\CommentVarTok}[1]{\textcolor[rgb]{0.56,0.35,0.01}{\textbf{\textit{#1}}}}
\newcommand{\ConstantTok}[1]{\textcolor[rgb]{0.56,0.35,0.01}{#1}}
\newcommand{\ControlFlowTok}[1]{\textcolor[rgb]{0.13,0.29,0.53}{\textbf{#1}}}
\newcommand{\DataTypeTok}[1]{\textcolor[rgb]{0.13,0.29,0.53}{#1}}
\newcommand{\DecValTok}[1]{\textcolor[rgb]{0.00,0.00,0.81}{#1}}
\newcommand{\DocumentationTok}[1]{\textcolor[rgb]{0.56,0.35,0.01}{\textbf{\textit{#1}}}}
\newcommand{\ErrorTok}[1]{\textcolor[rgb]{0.64,0.00,0.00}{\textbf{#1}}}
\newcommand{\ExtensionTok}[1]{#1}
\newcommand{\FloatTok}[1]{\textcolor[rgb]{0.00,0.00,0.81}{#1}}
\newcommand{\FunctionTok}[1]{\textcolor[rgb]{0.13,0.29,0.53}{\textbf{#1}}}
\newcommand{\ImportTok}[1]{#1}
\newcommand{\InformationTok}[1]{\textcolor[rgb]{0.56,0.35,0.01}{\textbf{\textit{#1}}}}
\newcommand{\KeywordTok}[1]{\textcolor[rgb]{0.13,0.29,0.53}{\textbf{#1}}}
\newcommand{\NormalTok}[1]{#1}
\newcommand{\OperatorTok}[1]{\textcolor[rgb]{0.81,0.36,0.00}{\textbf{#1}}}
\newcommand{\OtherTok}[1]{\textcolor[rgb]{0.56,0.35,0.01}{#1}}
\newcommand{\PreprocessorTok}[1]{\textcolor[rgb]{0.56,0.35,0.01}{\textit{#1}}}
\newcommand{\RegionMarkerTok}[1]{#1}
\newcommand{\SpecialCharTok}[1]{\textcolor[rgb]{0.81,0.36,0.00}{\textbf{#1}}}
\newcommand{\SpecialStringTok}[1]{\textcolor[rgb]{0.31,0.60,0.02}{#1}}
\newcommand{\StringTok}[1]{\textcolor[rgb]{0.31,0.60,0.02}{#1}}
\newcommand{\VariableTok}[1]{\textcolor[rgb]{0.00,0.00,0.00}{#1}}
\newcommand{\VerbatimStringTok}[1]{\textcolor[rgb]{0.31,0.60,0.02}{#1}}
\newcommand{\WarningTok}[1]{\textcolor[rgb]{0.56,0.35,0.01}{\textbf{\textit{#1}}}}
\usepackage{graphicx}
\makeatletter
\def\maxwidth{\ifdim\Gin@nat@width>\linewidth\linewidth\else\Gin@nat@width\fi}
\def\maxheight{\ifdim\Gin@nat@height>\textheight\textheight\else\Gin@nat@height\fi}
\makeatother
% Scale images if necessary, so that they will not overflow the page
% margins by default, and it is still possible to overwrite the defaults
% using explicit options in \includegraphics[width, height, ...]{}
\setkeys{Gin}{width=\maxwidth,height=\maxheight,keepaspectratio}
% Set default figure placement to htbp
\makeatletter
\def\fps@figure{htbp}
\makeatother
\setlength{\emergencystretch}{3em} % prevent overfull lines
\providecommand{\tightlist}{%
  \setlength{\itemsep}{0pt}\setlength{\parskip}{0pt}}
\setcounter{secnumdepth}{-\maxdimen} % remove section numbering
\ifLuaTeX
  \usepackage{selnolig}  % disable illegal ligatures
\fi
\IfFileExists{bookmark.sty}{\usepackage{bookmark}}{\usepackage{hyperref}}
\IfFileExists{xurl.sty}{\usepackage{xurl}}{} % add URL line breaks if available
\urlstyle{same}
\hypersetup{
  pdftitle={Working with data},
  hidelinks,
  pdfcreator={LaTeX via pandoc}}

\title{Working with data}
\author{}
\date{\vspace{-2.5em}}

\begin{document}
\maketitle

\n
\n

\hypertarget{introduction}{%
\subsection{Introduction}\label{introduction}}

The aim of this section is to explore some of the simple ways in which
you can read data into R.

There are many packages and functions available for reading in data.
I'll introduce two methods that can be used with common file types. Both
methods are straightforward to use.

The package that you use to read files into R really depends on the type
of file that you are trying to read in.

If you are working with Microsoft Excel files, then using
\texttt{read\_excel} is a nice, simple way of reading in files.

In contrast, \texttt{read.csv()} lets you read in files where columns
are separated, for example, by tabs, spaces, or commas. Files you might
want to open in this way are those ending in \texttt{.txt} and
\texttt{.csv}.

If you are working with more specialist file extensions, you will almost
definitely find an available package and documentation to help you work
with those files online via a quick Google search.

\n
\n

\hypertarget{working-directory}{%
\subsection{Working directory}\label{working-directory}}

Before thinking about loading files, it is worth considering where you
are loading files from.

The working directory is used to tell R where to look for files and
where to save them. Knowing your working directory can help you save
time when it comes to typing out file names to load into R. For example,
if you are loading a file from the working directory, you can simply
type the name of the file, rather than having to type out the entire
directory path, meaning:

\texttt{read.csv("file.csv")} versus something more cumbersome like
\texttt{read.csv("c:/Users/Me/Documents/file.csv")}

To find out your current working directory, type \texttt{getwd()} into
your console and press return.

To change your working directory, type \texttt{setwd()}, making sure to
put the new directory inside the brackets. For example,
\texttt{setwd("/Users/Me/Desktop")}. \emph{Note the quotation marks
around the filename -- the file won't be found otherwise.}

\n
\n

\hypertarget{read.csv}{%
\subsection{\texorpdfstring{\texttt{read.csv()}}{read.csv()}}\label{read.csv}}

\hypertarget{reading-in-.csv-and-.txt-files.}{%
\subsubsection{\texorpdfstring{Reading in \texttt{.csv} and
\texttt{.txt}
files.}{Reading in .csv and .txt files.}}\label{reading-in-.csv-and-.txt-files.}}

When you start up R, the \texttt{utils} package is automatically loaded,
meaning you don't need to do anything. You can use this package to
import data stored in the CSV format (``comma-separated values'' files,
ending .csv) using the \texttt{read.csv()} function.

To use this function, put the name of the file that you want to open
inside the brackets \texttt{read.csv("put\_me\_here\_please.csv")}. You
can put the file name inside single
\texttt{\textquotesingle{}\ \textquotesingle{}} or double \texttt{"\ "}
quotation marks. It doesn't matter which.

You can assign a name to the file that you are reading in by typing
\texttt{\textless{}-}.

For example typing

\begin{Shaded}
\begin{Highlighting}[]
\NormalTok{file }\OtherTok{\textless{}{-}} \FunctionTok{read.csv}\NormalTok{(}\StringTok{"Vowels\_to\_plot.xlsx"}\NormalTok{)}
\end{Highlighting}
\end{Shaded}

\begin{verbatim}
## Warning in read.table(file = file, header = header, sep = sep, quote = quote, :
## line 1 appears to contain embedded nulls
\end{verbatim}

\begin{verbatim}
## Warning in read.table(file = file, header = header, sep = sep, quote = quote, :
## line 2 appears to contain embedded nulls
\end{verbatim}

\begin{verbatim}
## Warning in read.table(file = file, header = header, sep = sep, quote = quote, :
## line 4 appears to contain embedded nulls
\end{verbatim}

\begin{verbatim}
## Warning in read.table(file = file, header = header, sep = sep, quote = quote, :
## line 5 appears to contain embedded nulls
\end{verbatim}

\begin{verbatim}
## Warning in scan(file = file, what = what, sep = sep, quote = quote, dec = dec,
## : embedded nul(s) found in input
\end{verbatim}

\begin{Shaded}
\begin{Highlighting}[]
\FunctionTok{print}\NormalTok{(file)}
\end{Highlighting}
\end{Shaded}

\begin{verbatim}
##                                                                                                                                                                                                                                                                                                                                                                                                                                                                                                                                                                                                                                                                                                                                                                                                                                                                                                         PK...
## 1                                                                                                                                                                                                                                                                                                                                                                                                                                                                                                                                                                                                                                                                                                                                                                                                                                                                                                          \a
## 2                                                                                                                                                                                                                                                                                                                                                                                                                                                                                                                                                                                                                                                                                                                                                           \x9f\x97\021\xb0\xe0j\x8f\x95h\x88⥔X7\xe0\024\x96!\x82\xe7\x95qHN\021?\xa6\x89\x8c\xaa\x9e\xaa\t\xc8A\xbf\177.\xeb\xe0\t<\xf5\xa8\xc5\020\xa3\xe1
## 3                                                                                                                                                                                                                                                                                                                                                                                                                                                                                                                                                                                                                                                                                                                                                                                       \x8c\xd5\xccRq\xbb\xe0\xd7+%oƋ\xe2z\xb5\xaf\xa5\xaa\x84\x8aњZ\021\v\x95s\xaf\177\x90\xf4\xc2xljС\x9e9\x86.1&P\032\033
## 4                                                                                                                                                                                                                                                                                                                                                                                                                                                                                                                                                                                                                                                                                                                                                                                                                                                                                         g\xff\xad\x81\xc7[n
## 5                                                                                                                                                                                                                                                                                                                                                                                                                                                                                                                                                                                                                                                                                                                                                                                                                                                               z_\xeeU\xa7\016\024\xf6\006r\x99\177f\xa3\017
## 6                                                                                                                                                                                                                                                                                                                                                                                                                                                                                                                                                                                                                                                                                                                                                                                                                                                                                \xbbg\xf6\xefx^\x9e\xa0~\001
## 7                                                                                                                                                                                                                                                                                                              JU\xf8\xaf>\x81\xfe\xa6#\x8a\xb602\xa4\x95]`\x89X\033R\xf6xb\xc4\xc9L\xb6\xfc+\xa0\xd57 /\x9e={\xfe\xf0\xe9\xf3\x87\xbf=\177\xf4\xe8\xf9\xc3_\xb2\xb9\xb5*Kn\a%SS\xeeՏ_\xff\xfd\xfd\027\xde_\xbf\xfe\xf0\xea\xf17\xe9\xd4'\xf1\xc2Ŀ\xfc\xf9˗\xbf\xff\xf1:\xf5\xb0\xe2\xdc\025/\xbe}\xf2\xf2\xe9\x93\027\xdf}\xf5\xe7O\x8f\035\xda\xdb\034\035\x98\xf0!\x89\xb1\xf0\xae\xe1c\xef&\x8ba\x81\016\xfb\xf1\001?\x9d\xc40BĒ@\021\xe8v\xa8\xee\xcb\xc8\002^[ \xea\xc2u\xb0\xed\xc2\xdb\034X\xc6\005\xbc<\xbfkٺ\037\xf1\xb9$\x8e\x99\xafF\xb1\005\xdcc\x8cv\030w:ચ\xcb\xf0\xf0p\x9eLݓ󹉻\x89Бk\xee.J\xac
## 8                                                                                                                                                                                                                                                                                                                                                                                                                                                                                                                                                                                                                                       \x8a\022\x89\xa68\xc1\xd2S\xbf\xb1C\x8c\035\xab\xbbC\x88\xe5\xd7=2\xe2L\xb0\x89\xf4\xee\020\xaf\x83\x88\xd3%Cr`%R.\xb4Cb\x88\xcb\xc2e \x84\xda\xf2\xcd\xdem\xafèk\xd5=|d#a[ \xea0~\x88\xa9\xe5\xc6\xcbh.Q\xecR9D15\035\xbe\x8bd\xe42r\177\xc1G&\xae/$Dz\x8a)\xf3\xfac
## 9                                                                                                                                                                                                                                                                                                                                                                                                                                                                                                                                                                                                                                                                                                                                                                                                                                                                                 \x84K\xe6:\x87\xf5\032A\xbf
## 10                                                                                                                                                                                                                                                                                              \f\xe3\016\xfb\036]\xc46\x92Kr\xe8ҹ\x8b\0303\x91=v؍P<s\xdaL\x92\xc8\xc4~&\016!E\x91w\x83I\027|\x8f\xd9;D=C\034P\xb21ܷ\t\xb6\xc2\xfdf\xb8\005\xe4j\x9a\x94'\x88\xfae\xce\035\xb1\xbc\x8c\x99\xbd\037\027t\x82\xb0\x8be\xda<\xb6ص͉3;:\xf3\xa9\x95ڻ\030St\x8c\xc6\030{\xb7>sX\xd0a3\xcb\xe7\xb9\xd1W`\x95\035\xecJ\xac+\xc8\xceU\xf5\x9c`\001e\x92\xaak\xd6)r\x97\b+e\xf7\xf1\x94m\xb0goq\x82x\026(\x89\021ߤ\xf9\032D\xddJ]8\xe5\x9cTz\x9d\x8e\016M\xe05\002\xe5\037\xe4\x8b\xd3)\xd7\005\xe80\x92\xbb\xbfI\xeb\x8d\bYg\x97z\026\xee|]p+~o\xb3\xc7`_\xde=\xed\xbe\004\031|j\031 \xf6\xb7\xf6\xcd\020Qk\x82<a\x86\b
## 11                                                                                                                                                                                                                                                                                                                                                                                                                                                                                                                                                                                                                                                                                                                                                   \f\027݂\x88\025\xfe\\D\x9d\xabZl؛6\017\003\024FV\xbd\023\x93\xe4\x8d\xc5ω\xb2'\xfcw\xca\036w\001s\006\005\x8f[\xf1\xfb\x94:\x9b(e\xe7D\x81\xb3\t\xf7\037
## 12 kzh\x9e\xdc\xc0p\x92\xacs\xd6yUs^\xd5\xf8\xff\xfb\xaaf\xd3^>\xafe\xcek\x99\xf3Z\xc6\xf5\xf6\xf5Aj\x99\xbc|\x81\xca&\xef\xf2\xe8\x9eO\xbc\xb1\xe53!\x94\xee\xcb\005ŻBw}\004\xbcь\a0\xa8\xdbQ\xba'\xb9j\001\xce\xf8\x9a5\x98,ܔ#-\xe3q&?'2ڏ\xd0\fZCe\xdd\xc0\x9c\x8aL\xf5Tx3&\xa0c\xa4\x87u+\025\x9fЭ\xfbN\xf3x\x8f\x8d\xd3Ng\xb9\xac\xba\x9a\xa9\v\005\x92\xf9x)\\\x8dC\x97J\xa6\xe8Z=\xefޭ\xd4\xeb~\xe8TwY\x97\006(\xd9\xd3\030aLf\033Qu\030Q_\016B\024^g\x84^ٙX\xd1tX\xd1PꗡZFq\xe5\n0m\025\025x\xe5\xf6\xe0E\xbd\xe5\x87A\xdaA\x86f\034\x94\xe7c\025\xa7\xb4\x99\xbc\x8c\xae\nΙFz\x933\xa9\x99\001Pb/3 \x8ftSٺqyjui\xaa\xbdE\xa4-#\x8ct\xb3\x8d0\xd20\x82\027\xe1,;͖\xfbYƺ\x99\x87\xd42O\xb9b\xb9\033r3\xea\x8d\017\021kE'\xb8\x81&&S\xd0\xc4;n\xf9\xb5j\b\xb7*#4k\xf9\023\xe8\030\xc3\xd7x\006\xb9#\xd4[\027\xa2S\xb8v\031I\x9en\xf8wa\x96\031\027\xb2\x87D\x94:\\\x93N\xca\0061\x91\x98{\x94\xc4-_-\177\x95
## 13                                                                                                                                                                                                                                                                                                                                                                                                                                                                           4\xd1\034\xa2m+W\x80\020>Z\xe3\x9a@+\037\x9bq\020t;\xc8x2\xc1#i\x86\xdd\030Q\x9eN\037\x81\xe1S\xaep\xfe\xaa\xc5\xdf\035\xac$\xd9\034½\037\x8d\x8f\xbd\003:\xe77\021\xa4XX/+\a\x8e\x89\x80\x8b\x83r\xea\xcd1\x81\x9b\xb0\025\x91\xe5\xf9w\xe2`\xcah\u05fc\x8a\xd29\x94\x8e#:\x8bPv\xa2\x98d\x9e\xc25\x89\xae\xcc\xd1O+\037\030Oٚ\xc1\xa1\xeb.<\x98\xaa\003\xf6\xbdO\xdd7\037\xd5\xcas\006i\xe6g\xa6\xc5*\xea\xd4t\x93\xe9\x87;\xe4
## 14                                                                                                                                                                                                                                                                                                                                                                                                                                                                                                                                                                                                                                                                                                                                                                                                     \xab\xf2CԲ*\xa5n\xfdN-r\xaek.\xb9\016\022\xd5yJ\xbc\xe1\xd4}\x8b\003\xc10-\x9f\xcc2MY\xbcNÊ\xb3\xb3Q۴3
## 15                                                                                                                                                                                                                                                                                                                                                                                                                                                                                                                                                                                                                                                                                                                                                                                                                                                                      \b\fO\xd46\xf8muF8=\xf1\xae'?ȝ\xccZu@
## 16                                                                                                                                                                                                                                                                                                                                                                                                                                                                                                             \xebJ\x9d\xf8\xfa\xcaܼ\xd5f\aw\x81<zp\1778\xa7R\xe8PBo\x97#(\xfa\xd2\033Ȕ6`\x8bܓY\x8d\b\u07fc9'-\xff~)l\a\xddJ\xd8-\x94\032a\xbf\020T\x83R\xa1\021\xb6\xab\x85v\030V\xcb\xfd\xb0\\\xeau*\017\xe0`\x91Q\\\016\xd3\xeb\xfa\001\\a\xd0Evi\xaf\xc7\xd7.\xee\xe3\xe5-ͅ\021\x8b\x8bL_\xcc\027\xb5\xe1\xfa\xe2\xbe\\\xd9|q\xef\021 \x9d\xfb\xb5ʠYmvj\x85f\xb5=(\004\xbdN\xa3\xd0\xec\xd6:\x85^\xad[\xef
## 17                                                                                                                                                                                                                                                                                                                                                                                                                                                                                                                                                                                                                                                                                                                                                                                                                                                                                       zݰ\xd1\034<\xf0\xbd#
## 18                                                                                                                                                                                                                                                                                                                                                                                                                                                                                                                                                                                                                                                                                                                                                                                                                                                                                        \016\xda\xd5nP\xeb7
## 19                                                                                                                                                                                                                                                                                                                                                                                                                                                                                                                                                                                                                                                                                                                                                                                                                                   \xb5r\xb7[\bj%e~\xa3Y\xa8\a\x95J;\xa8\xb7\033\xfd\xa0\xfd +c`\xe5)}d\xbe
## 20                                                                                                                                                                                                                                                                                                                                                                                                                                                                                                                                                                                                                                                                                                                      \xd0D\x98\xe4i\xa9\x856\xc8A\xeb\xa0r\x9dFQ\xc9z\x8f[*\xf8\xc6p\xefVS\xc9űW\xc7^\xd1u{\xf0\x93\034j\xef\x95\xc4\xf3\xe8ٜ\xe3̼\xe5\xed\xa0\035\xb6\005p.\xc4(\xd5^\x91\xa7\xf0&\0343\xaa
## 21                                                                                                                                                                                                                                                                                                                                                                                                                                                                                                                                                                                                                                                                                                                                                                             \veT\x9fS\xa8\xa2\022\xcd\a\xc9 1\xa6@p\u05f6\xf7\x90\xc9pBC\xf8U\xdc\027s\xd3\xf4\003\xad\xeb\xea׃\x97:o\xd9\036^ϧ\x97O{\x97X
## 22                                                                                                                                                                                                                                                                                                                                                                                                                                                                                                                                                                                                                                                                                                                                                                                                9\xd0wU\xc2\xff\xd6\023G)h\xcb\xe6=\xc03Vc\xaaO\xc2V\xe5\xa2n\xec'\t\xd7Llgd\x9ci\xbb`\xec\xe1\xef\xb7\xe3;
## 23                                                                                                                                                                                                                                                                                                                                                                                                                                                                                      \f\025Ԡ\xbc\xc34\xa1\xf8\xc7\xeb\xc1\xd6\xee\xcf\v\x9d2r֥?\x99\xd0\xe9\034w̖\xa2\027\a\xf7ѕ\x83\xb1i\x9a\xa4\x99t1B~\x8a\xdf֫MW5.U\xbb+\001\xa8Ȥ`\xc2\002\xf7\xda\026\033\xa1\xbd\x8fVД.ãy\xbbÊ;\xbf\016\xebޕ \037N\xbf\xad\xd7r\xa0v%z4\xc8(\xc4b}\x89\x8b\xf2:y|\xda.Q\x91\x92\x94Ɣ\xc6\xe9\xfd\x96\xccY:c\023\xf2\u07be\xfe\xeb~\033\xb3\037\xd4\xe7\f\xff\023\xa71Ic2\xdfR\xc2\xe8\x94Mg#\xe2\005Pd\xf8\xeak\024\xdf
## 24                                                                                                                                                                                                                                                                                                                                                                                                                                                                                                                                                        .x\xac\xc4\021I\xdc\xe9\x8bOj\x93B\xc4\xc4\026\xa9\xc8\026\x9e*\xd12Ǖ\x94dZ\xec\x80\026\xb9\xeds\xa7\t\xa9\003\xcei\xda\xcb\xd04\xd6\xe0}0o\035z\x96Wey+\xf1\xc0\xe8k\xac/\xe3\xc9PL\x8e\xab\x9e\xff״\016fࣗ\xdd1f`\xad\xbe\xc4\xe8\xac\001ί\xd4O֤@\xa1\xe1\xe2\t\x8c\xf5\034\xa8-\xbe\036\f:%\xe72\x959\xb7hޒ\xe5\xa3.\x95\x9c\xa7jk\xc0\xe1:\x8f\xd0
## 25                                                                                                                                                                                                                                                                                                                                                                                                                                                                                                                                                                                                                                                                                                                                                                                                                                                                                                       8B%?
## 26                                                                                                                                                                                                                                                                                                                                                                                                                                       \xea\021aX\xdf\006l\xadz^\xf5h8\xa4\x82쯼\xc0+Q\xfc\004\xc2\001\xac\022=$\v\x9e3\xe0 \x9b\x921v\x918\xe9\037!\xbdR\x8bȤd\026L\xc51\x9ck籽\xd1\xcbQ\x90\x83s\xe1`0\x81\xe4\xc69\xe2βC\xfa\xdel \xf1?\x88\x97s\xe2\x91a\xe2\x9dp\xb6\003\xdf4s\xce7>9O\xfa\xc3{\035\xba\b\xfe\x98\033\xa7\xe8\x9b\xf5\xaf\xf4\034w\xe1\036\030\xdf\xd7y^T\xdb\026\022\xd6\xf9\aN\xeb>\025\xd4c\xdedr\x83ɺ\005\xbf\xc7\xfa]\xf3wc8\x83\x97\xe9\xd6\xf5\xf2vQ^\x97\xf9_g5%?\xaeZ\xff\006
## 27                                                                                                                                                                                                                                                                                                                                                                                                                                                                                                                                                                                                                                                                                                                                                                                                         \x89&\x90ki\xa1\xd7e\035セ#\xbbkh)\xfd\xef]\xe9)\xedix\xf3\x98\xf7=\xa6\xda\177\xe8)\xba\xa1u\x8aL
## 28                                                                                                                                                                                                                                                                                                                                                                                                                                                                                                                                                                                                                                                                                                                                                                                                                                                                                                      \x8fI
## 29                                                                                                                                                                                                                                                                                                                                                                                                                            \021\032I\xbd2\xd7\032\xde^\xcfq\001\x91\xf3\xc2\xf4b\x835\030\x82}\xb3y\xa8z\xb7\xeb\x85\027ΓŋG\035\x85\x85\n\xf3\xd2\xd5\xf0u\xe2\xed\xb1\xcc\xf2,.\x8b\xae\x8c\xb7\x9cwqqjy\xfc\xbc\xcd\xcf9o\xdbÑ\xb7\xdf\020\005\xb4\t1\xae\x86\xd1\xfbyǘ\x93#j\xe1\022\x9a\xd1\004s \xab\x85\017\xd2^\031\n\x83\x92ؑ\\4\032Ϟ\xd24gr\tx\xfd\xae'h\xd6>\xbf\xd7/8\xb8{\xb9V[\xac\xfaG\xd1JZr4\xf8D\x92fn\024\026gR!\xfc\x961I\xc6\a\x8e\xff\x9c\x91\xad5\034\xb0\xa6b\177 \xab\xbe{B\xf3\003
## 30                                                                                                                                                                                                                                                                                                                                                                                                                                                                                                                                                                                                                                                                                                                                                                                                                                                                                               3{\x8a\0066U
## 31                                                                                                                                                                                                                                                                                                                                                                                                                                                                                                                                 \xa3\xa3\xde\xc7\xd1\xc0\xf9\xf4\xf5\xb1\003\xc5bcog\x8ah\xe0\x81\f\x87v\xbdj~p\xb6R\x96x\xf2\x89UQ\033\x98D\xd2^kv\023\006\xcb\025%\x8ce2P\016VJ\x99G\x9d\xac\xbb\xda\021\xf5\xb6\xae?u~5\xa0}3U\xd7\033\xc8]\xbf\001uz\xa4\x92\xfc\xbfM\xc3\xe0\035\036\xc9-\001\xa3\xfc\021\xa1\xdd\xc2B\xe1\022\xe6\xefL\x89\x8bl\xf3\x88b\xc0\v\x86gk[\x95{A\xb7\x8d~\xfb\xaf\xfd\005
## 32                                                                                                                                                                                                                                                                                                                                                                                                                                                                                                                                                                                                                                                                                                                                                                                                                                                                                                  \xce\xddo
## 33                                                                                                                                                                                                                                                                                                                                                                                                                                                                                                                                                                                                                                                                                                                                                                          \037ȽPgd61\xf5\x919\xb0\xccˋ\xc1x\xa2\x9b\x83\x99mL\006\027H\x9f\022k2\xb1'd\x96\xca\b\xfb[\xa8#b#\x8c\xb11\x98\x8du4\030\x9b\030
## 34                                                                                                                                                                                                                                                                                                                                                                                                                                                                                                                                                                                                                                                                                                                                                                                     f`\\\003c\xb2\x99\032csb\x99\x9b\xcb\xec\xb8\034/\xa0L(~~P\xad\xd6\xd3\xd2\xe3\xaeӷZ>\xa3'.\x91\xfe\032\t\xb0P\vG\x9e.
## 35                                                                                                                                                                                                                                                                                                                                                                                                                                                                                                                                                                                                                                                                                                                                                                                                                                                                                              #\xf6B\x95&sK
## 36                                                                                                                                                                                                                                                                                                                                                                                                                                                                                                                                                                                                                                                                                                                                                                                                                                                                                             \a\xdd\021\xf6
## 37                                                                                                                                                                                                                                                                                                                                                                                                                                                                                                                                                                                                                                                                                      \x8b@͚1\002g#\xccG\xee\tNP\025Z\x9f}\xf7)\xa1̌\0332\xa7K\x8ai\x8d\021\016\xd5\024\x96AS\xd9 \0162\x83\021\xcd?QA\xe24\023\xbbU\x89\xac\xeay\xd2Q\xcbٰ\xb8\xc1J\032\xca\xf4t\xd9MO56g\xe9\xaa\xc6\xed\xe5諐u3]\031аt1\xaa
## 38                                                                                                                                                                                                                                                                                                                                                                                                                                                                                                                                                                                                                                                                                                                                                                                                                                            \x8f&\x8b)\xbdv\xa6\006\xd6d/\x9e\xc9b\x8a\xa5C\xae\002\xa3\x9b
## 39                                                                                                                                                                                                                                                                                                                                                                                                                                                                                                                                                                                                                                                                                                                                                          \xcaL:\xb8Q\x99\xf0\xe5h%.\x89r}\x8c\xba\xe9#'o\xb2\x86\xf4\x8d\xe7p\x854\xdexh\xff\x91\xf8[\xb1S\036\x90\033Bp\031M&\005+)\004\x9f<\x85\x9f\021k
## 40                                                                                                                                                                                                                                                                                                                                       \xe5_\xae(蟻\xf8N\x85G\x93\xb6J\xaf\xbd\034+\xa9U\x86\xb9\036\xa6\xdd\xf4p\x80S\a/:@\xfdr\x94T\xad}s\035uL\xdeuF\035TT'~9\032\xaa\xd6\xf6\xb9\x86:\xa6\xed:\xa3\016\032\xaa\023\xbf\x9c4]\xec]2\xed\x8cZ\xe4\xe9*aS\x80)\xbf\xf7\xa2\x83r\\#\xb6\xea\xa2Z\xe14\xca\xff]\xd4\177\xbd\x8b\xaa\xe3.y\x88\031\x9fNT1\xb9q\x9b\xa8M*wE\x87\x9878\xd0\001p.y$\033X\x80H2\x9cn\017\x9f\xe4\xfc\xa0\xd0%I\xc8\xe8\026Z|'ǡ\x8f\x82\x84\xf2\xfd\033h\x96+\x80\n\xcc\026\xf6)\xb8\027\xfa\033\032B#i\035\x83\017\x8b(G\xbd\xf9Z学\xaaȆ~\xa1F\0209\xb4y\xfe\xf6\024V$!مz\x851H!n
## 41                                                                                                                                                                                                                                                                                                                                                                                                                                                                                                                                                                                                                                                                                                                                                                                                                                        \xff\xd6;\x87\xc7x\xa2\002\xf6\xe5H\025qE\xec\x88\xe2\x87ކ0\x85\xda
## 42                                                                                                                                                                                                                                                                                                                      G\017\xf0\x8c2%\025\x92\017\x955\xbc\x81\x82\xc0\x95\004\xb2\x8e\a&\x80?\006\024\xaaz\x98\xd5(\x90k\x950\x80\x8b+\xb2\xcc\020\xf9\022\xc8\006OW\b»\x8c\xd9\xf0\xd5O\x87D\x8bG\x90\xd5]\xc4OQi\xb7\xa7'(\xae\034\xde\xc3\005%96|w\xf6\xe9G\xf7\003Ґ\xe5%l{\x99c\xdc\024\xdd\xc1\xc1\x85\xbd\xfc\xa3\xd5T7\xce9\xb1Y\x9cu\xb5\xe7\xf8\x80\xef\024\xccТVz\xceAi2umu\xe3'\xd1M9\x8b~\x81\v\f\xe0\x8e\xc9w\xb5\xcev^n\033\xb3\x91]d\034\001\x9e\x97\xfa\xc6R\xd8}#m8}\xbf\xb2tJ]h\031\xfb\xd2\xc6\xdd\xe1qj)He\xc8\017\x97\xf6J\x97u\x9ae\x8d\xc3\xfb\xc1\xfex.\xa7\xa0\xd0\036\037\xdbjCo
## 43                                                                                                                                                                                                                                                                                                                                                                                                                                                                                                                                                                                                                                                                                                                                                                                                                        %\xeaH\\\xee\x83ۓ\xbf[-\xaf8\xa7؁\x90g\xbd\x87\xdaA<\xf5>n\xe0\x95phα\0054\xa2j\002Ճ
## 44                                                                                                                                                                                                                                                                                                                                                                                                                                                                                                                                                                                                                                                                                                                                                                                                      U\x92A\x94U\xba\xfbY\xb4v\025Z\xa8\005\x95g\xa8\x88r\031\xfb\x9do>'8R\vErf3B\xed\xd0\037[\xbc\xf9\033
## 45                                                                                                                                                                                                                                                                                                                                                                                                                                                                                                                                                                                                                                                                                                                                                                                                                                                                                                         \a
\end{verbatim}

\begin{Shaded}
\begin{Highlighting}[]
\FunctionTok{library}\NormalTok{(readxl)}
\NormalTok{file }\OtherTok{\textless{}{-}} \FunctionTok{read\_excel}\NormalTok{(}\StringTok{"Vowels\_to\_plot.xlsx"}\NormalTok{)}
\end{Highlighting}
\end{Shaded}

lets you read in a file called
\texttt{\textquotesingle{}Vowels\_to\_plot.xlsx\textquotesingle{}}
(assuming that the file actually exists in your working directory). The
file is then available to work with under the name \texttt{file}.

The file would look like this:

\begin{verbatim}
## # A tibble: 18 x 4
##    Speaker   Vowel    F1    F2
##    <chr>     <chr> <dbl> <dbl>
##  1 Example   i       240  2400
##  2 Example   e       390  2300
##  3 Example   ɛ       610  1900
##  4 Example   æ       631  1355
##  5 Example   ɑ       750   940
##  6 Example   ɒ       700   760
##  7 Example   ɔ       500   700
##  8 Example   ʊ       419  1063
##  9 Example   u       250   595
## 10 Example 2 i       300  2500
## 11 Example 2 e       490  2400
## 12 Example 2 ɛ       710  2000
## 13 Example 2 æ       721  1445
## 14 Example 2 ɑ       850  1030
## 15 Example 2 ɒ       800   855
## 16 Example 2 ɔ       600   799
## 17 Example 2 ʊ       520  1170
## 18 Example 2 u       325   699
\end{verbatim}

To view the file in R, type \texttt{View()}, inserting the name of the
file in the brackets, e.g:

\texttt{View(file)}

\emph{Note the absence of quotation marks}

\n

\hypertarget{aside}{%
\paragraph{Aside}\label{aside}}

There could be several reasons why a file might not load. One reason is
that the file name might contain a typo. Another reason could be that
the file extension is not correct. If it is still not loading, try
typing out the entire path of the file (it could be that the file isn't
in your working directory), making sure to use forward slashes
(\texttt{/}) and not back slashes (\texttt{\textbackslash{}}).

E.g. \texttt{read.csv("c:/Users/SteveIrwin/Documents/animaldata.csv")}

\hypertarget{additional-parameters-for-read.csv}{%
\subsubsection{\texorpdfstring{Additional parameters for
\texttt{read.csv()}}{Additional parameters for read.csv()}}\label{additional-parameters-for-read.csv}}

\hypertarget{reading-in-non-comma-separated-files}{%
\subparagraph{Reading in non comma separated
files}\label{reading-in-non-comma-separated-files}}

However, it is not just possible to open \texttt{.csv} files. If your
data are stored in \texttt{.txt} files, these can also be accessed using
the \texttt{read.csv()} function. To do this, you simply have to adjust
the \texttt{sep} parameter when opening the file.

If you are opening a file and your columns are separated by tabs, for
example, you just need to add \texttt{sep\ =\ "\textbackslash{}t"} when
opening the file. For data separated by spaces/whitespace, simply add
\texttt{sep\ =\ ""} instead. The default \texttt{sep} setting is
\texttt{,}, so you don't need to type anything for it if reading in a
comma separated file, as was shown above.

\begin{Shaded}
\begin{Highlighting}[]
\FunctionTok{read.csv}\NormalTok{(}\StringTok{"Vowels\_to\_plot.txt"}\NormalTok{, }\AttributeTok{sep =} \StringTok{"}\SpecialCharTok{\textbackslash{}t}\StringTok{"}\NormalTok{)}
\end{Highlighting}
\end{Shaded}

\begin{verbatim}
##      Speaker Vowel  F1   F2
## 1    Example     i 240 2400
## 2    Example     e 390 2300
## 3    Example     ɛ 610 1900
## 4    Example     æ 631 1355
## 5    Example    ɑ  750  940
## 6    Example     ɒ 700  760
## 7    Example     ɔ 500  700
## 8    Example     ʊ 419 1063
## 9    Example     u 250  595
## 10 Example 2     i 300 2500
## 11 Example 2     e 490 2400
## 12 Example 2     ɛ 710 2000
## 13 Example 2     æ 721 1445
## 14 Example 2    ɑ  850 1030
## 15 Example 2     ɒ 800  855
## 16 Example 2     ɔ 600  799
## 17 Example 2     ʊ 520 1170
## 18 Example 2     u 325  699
\end{verbatim}

\begin{verbatim}
## 'data.frame':    18 obs. of  4 variables:
##  $ Speaker: chr  "Example" "Example" "Example" "Example" ...
##  $ Vowel  : chr  "i" "e" "ɛ" "æ" ...
##  $ F1     : int  240 390 610 631 750 700 500 419 250 300 ...
##  $ F2     : int  2400 2300 1900 1355 940 760 700 1063 595 2500 ...
\end{verbatim}

The structure of the data can be obtained by typing \texttt{str(file)},
where \texttt{file} is replaced by whatever you called your data. By
typing this you can see whether R has read your data in as characters
(\texttt{chr}), numbers (e.g.~\texttt{int} or \texttt{num}) or factors
(\texttt{Factor}), for example.

\hypertarget{setting-strings-as-factors}{%
\subparagraph{Setting strings as
factors}\label{setting-strings-as-factors}}

With \texttt{stringsAsFactors}, you can tell R whether it should convert
strings (characters/letters or combinations of letters and numbers,
e.g.~`cat' `File001') to factors. \texttt{stringsAsFactors} used to be
set to \texttt{TRUE} by default, but it now appears to be set to
\texttt{FALSE}.

If you wish to read in strings as factors, simply add
\texttt{stringsAsFactors\ =\ TRUE} after the name of the file that you
are reading in, like so:

\begin{Shaded}
\begin{Highlighting}[]
\NormalTok{file3 }\OtherTok{\textless{}{-}} \FunctionTok{read.csv}\NormalTok{(}\StringTok{\textquotesingle{}Vowels\_to\_plot.txt\textquotesingle{}}\NormalTok{, }\AttributeTok{sep =} \StringTok{"}\SpecialCharTok{\textbackslash{}t}\StringTok{"}\NormalTok{, }\AttributeTok{stringsAsFactors =} \ConstantTok{TRUE}\NormalTok{)}
\end{Highlighting}
\end{Shaded}

\begin{Shaded}
\begin{Highlighting}[]
\FunctionTok{str}\NormalTok{(file3)}
\end{Highlighting}
\end{Shaded}

\begin{verbatim}
## 'data.frame':    18 obs. of  4 variables:
##  $ Speaker: Factor w/ 2 levels "Example","Example 2": 1 1 1 1 1 1 1 1 1 2 ...
##  $ Vowel  : Factor w/ 9 levels "æ","ɑ ","ɒ","e",..: 6 4 5 1 2 3 7 9 8 6 ...
##  $ F1     : int  240 390 610 631 750 700 500 419 250 300 ...
##  $ F2     : int  2400 2300 1900 1355 940 760 700 1063 595 2500 ...
\end{verbatim}

As you can see, where \texttt{Animal} and \texttt{Name} were read in as
strings earlier, by typing \texttt{stringsAsFactors\ =\ TRUE}, they are
now read in as factors.

\hypertarget{setting-na-values}{%
\subparagraph{Setting NA values}\label{setting-na-values}}

An additional parameter that might come in handy is \texttt{na.strings}.
By default, \texttt{read.csv()} takes ``NA'' to stand for NA values in
the dataset. However, it is possible that you may have coded your data
differently, so that maybe `0', `na' or `N/A', for example are used to
represent NA values. You can specify this using the \texttt{na.strings}
parameter, typing, for example

\begin{Shaded}
\begin{Highlighting}[]
\NormalTok{file4 }\OtherTok{\textless{}{-}} \FunctionTok{read.csv}\NormalTok{(}\StringTok{\textquotesingle{}Vowels\_to\_plot.txt\textquotesingle{}}\NormalTok{, }\AttributeTok{sep =} \StringTok{"}\SpecialCharTok{\textbackslash{}t}\StringTok{"}\NormalTok{, }\AttributeTok{na.strings =} \StringTok{"i"}\NormalTok{)}
\FunctionTok{str}\NormalTok{(file4)}
\end{Highlighting}
\end{Shaded}

\begin{verbatim}
## 'data.frame':    18 obs. of  4 variables:
##  $ Speaker: chr  "Example" "Example" "Example" "Example" ...
##  $ Vowel  : chr  NA "e" "ɛ" "æ" ...
##  $ F1     : int  240 390 610 631 750 700 500 419 250 300 ...
##  $ F2     : int  2400 2300 1900 1355 940 760 700 1063 595 2500 ...
\end{verbatim}

\begin{Shaded}
\begin{Highlighting}[]
\NormalTok{file5 }\OtherTok{\textless{}{-}} \FunctionTok{read.csv}\NormalTok{(}\StringTok{\textquotesingle{}Vowels\_to\_plot.txt\textquotesingle{}}\NormalTok{, }\AttributeTok{sep =} \StringTok{"}\SpecialCharTok{\textbackslash{}t}\StringTok{"}\NormalTok{, }\AttributeTok{na.strings =} \FunctionTok{c}\NormalTok{(}\StringTok{"i"}\NormalTok{, }\StringTok{"e"}\NormalTok{, }\StringTok{"u"}\NormalTok{, }\StringTok{"750"}\NormalTok{))}
\FunctionTok{str}\NormalTok{(file5)}
\end{Highlighting}
\end{Shaded}

\begin{verbatim}
## 'data.frame':    18 obs. of  4 variables:
##  $ Speaker: chr  "Example" "Example" "Example" "Example" ...
##  $ Vowel  : chr  NA NA "ɛ" "æ" ...
##  $ F1     : int  240 390 610 631 NA 700 500 419 250 300 ...
##  $ F2     : int  2400 2300 1900 1355 940 760 700 1063 595 2500 ...
\end{verbatim}

\hypertarget{dont-use-the-first-line-as-headings}{%
\subparagraph{Don't use the first line as
headings}\label{dont-use-the-first-line-as-headings}}

By default, \texttt{read.csv()} will read the first line of your file in
as headings. If your data does not include headings, you can prevent
this by typing \texttt{header\ =\ FALSE}.

\begin{Shaded}
\begin{Highlighting}[]
\NormalTok{file6 }\OtherTok{\textless{}{-}} \FunctionTok{read.csv}\NormalTok{(}\StringTok{\textquotesingle{}Vowels\_to\_plot.txt\textquotesingle{}}\NormalTok{, }\AttributeTok{sep =} \StringTok{"}\SpecialCharTok{\textbackslash{}t}\StringTok{"}\NormalTok{, }\AttributeTok{header =} \ConstantTok{FALSE}\NormalTok{)}
\NormalTok{file6}
\end{Highlighting}
\end{Shaded}

\begin{verbatim}
##           V1    V2  V3   V4
## 1    Speaker Vowel  F1   F2
## 2    Example     i 240 2400
## 3    Example     e 390 2300
## 4    Example     ɛ 610 1900
## 5    Example     æ 631 1355
## 6    Example    ɑ  750  940
## 7    Example     ɒ 700  760
## 8    Example     ɔ 500  700
## 9    Example     ʊ 419 1063
## 10   Example     u 250  595
## 11 Example 2     i 300 2500
## 12 Example 2     e 490 2400
## 13 Example 2     ɛ 710 2000
## 14 Example 2     æ 721 1445
## 15 Example 2    ɑ  850 1030
## 16 Example 2     ɒ 800  855
## 17 Example 2     ɔ 600  799
## 18 Example 2     ʊ 520 1170
## 19 Example 2     u 325  699
\end{verbatim}

\emph{In the example above, no column names were provided. The first row
of the file contained only data. Having set \texttt{header} to
\texttt{FALSE}, R has automatically supplied \texttt{V1}, \texttt{V2}
etc. as headings.}

\hypertarget{trim-whitespace}{%
\subparagraph{Trim whitespace}\label{trim-whitespace}}

A very useful parameter that can be used is \texttt{strip.white}. By
setting \texttt{strip.white} to \texttt{TRUE} you can remove leading and
trailing spaces. If you do this, then
\texttt{\textquotesingle{}Tiger\textquotesingle{}},
\texttt{\textquotesingle{}\ Tiger\textquotesingle{}} and
\texttt{\textquotesingle{}Tiger\ \textquotesingle{}} will all be read in
as \texttt{\textquotesingle{}Tiger\textquotesingle{}}, rather than three
separate names -- something you probably don't want.

E.g.
\texttt{spaceless\ \textless{}-\ read.csv("animaldata.csv",\ strip.white\ =\ TRUE)}

\n
\n

\hypertarget{the-codeless-way}{%
\subsubsection{The codeless way}\label{the-codeless-way}}

However, there is another simple method for reading in csv/txt files, in
a very user friendly manner. In RStudio, you can click \texttt{file}
-\textgreater{} \texttt{Import\ Dataset} -\textgreater{}
\texttt{From\ Text\ (base)}. This lets you search through your files to
select the relevant one. You can choose then choose a name to give to
the file and you are presented with multiple parameters which you can
adjust as you need to.

\n
\n
\n

\hypertarget{read_excel}{%
\subsection{\texorpdfstring{\texttt{read\_excel()}}{read\_excel()}}\label{read_excel}}

\hypertarget{reading-in-excel-files}{%
\subsubsection{Reading in Excel files}\label{reading-in-excel-files}}

One package you can use to import Excel files is \texttt{readxl}. One
small caveat: *It can only read one sheet a time. If you need to work on
multiple sheets, they will each need to be loaded as objects.

To read in an Excel file, you first need to load the \texttt{readxl}
package, by typing the following into the console:

\texttt{library(readxl)} unless you have already loaded the
\texttt{tidyverse} package.

*If you don't have \texttt{tidyverse} or \texttt{readxl} packages
available, these can be obtained by typing
\texttt{install.packages(\textquotesingle{}tidyverse\textquotesingle{})}
or \texttt{install.packages(\textquotesingle{}readxl\textquotesingle{})}
respectively.

Following this, you can read in the file by typing
\texttt{read\_excel()}, with the name of the file surrounded by
quotation mark placed inside the brackets, as can be seen in the example
below:

\texttt{dataset\ \textless{}-\ read\_excel("Im\_a\_little\_dataset\_short\_and\_stout.xlsx")}

\hypertarget{additional-parameters-for-read_excel}{%
\subsubsection{\texorpdfstring{Additional parameters for
\texttt{read\_excel()}}{Additional parameters for read\_excel()}}\label{additional-parameters-for-read_excel}}

\hypertarget{trim-whitespace-1}{%
\subparagraph{Trim whitespace}\label{trim-whitespace-1}}

Another potentially useful parameter that is set to \texttt{TRUE} by
default is \texttt{trim\_ws}. This removes leading and trailing spaces.
What this means in practice is that
\texttt{\textquotesingle{}John\textquotesingle{}},
\texttt{\textquotesingle{}\ John\textquotesingle{}} and
\texttt{\textquotesingle{}John\ \textquotesingle{}} would all be read in
as \texttt{\textquotesingle{}John\textquotesingle{}}, which is
incredibly useful, especially if you can't see that you added a space at
the end of a variable. When it comes to conducting analyses, if it
weren't for the trimming of whitespace, the three Johns would count as
separate variables, rather than one, which is probably not what you
would want. If for some reason, you need to keep this whitespace, add
\texttt{trim\_ws\ =\ FALSE}, when reading in the file
(e.g.~\texttt{dataset\ \textless{}-\ read\_excel("Im\_a\_little\_dataset\_short\_and\_stout.xlsx,\ trim\_ws\ =\ FALSE")})

\hypertarget{setting-na-values-1}{%
\subparagraph{Setting NA values}\label{setting-na-values-1}}

As with \texttt{read.csv()}, you can set NA to be whatever you choose,
in this case using the \texttt{na} parameter. The default value for NA
in \texttt{read\_excel()} is a blank cell. You could add `na', `NA',
`0', or any variable or your choosing by typing something like

\texttt{dataset\_sheet2\ \textless{}-\ read\_excel("Im\_a\_little\_dataset\_short\_and\_stout.xlsx",\ sheet\ =\ 2,\ na\ =\ c("",\ "na",\ "NA",\ "woof\ woof"))}

\hypertarget{the-codeless-way-1}{%
\subsubsection{The codeless way}\label{the-codeless-way-1}}

As for \texttt{read.csv}, there is a very simple and codeless way of
reading in Excel files. In RStudio, you can click \texttt{file}
-\textgreater{} \texttt{Import\ Dataset} -\textgreater{}
\texttt{From\ Excel..}. You can then search for the file you want. You
are also presented with multiple parameters that you can adjust as you
wish.

\end{document}
